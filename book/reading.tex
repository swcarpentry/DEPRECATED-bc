\chapter{Recommended Reading}\label{s:bib}

\section{Books}

Susan A. Ambrose, Michael W. Bridges, Michele DiPietro, Marsha C.
Lovett, and Marie K. Norman:
\emph{How Learning Works: Seven Research-Based Principles for Smart Teaching}.
Jossey-Bass, 2010, 978-0470484104.
The best single-volume guide to evidence-based practices in education
around.

Chris Fehily: \emph{SQL: Visual QuickStart Guide} (3rd ed).
Peachpit Press, 0321553578, 2002.
Describes the 5\% of SQL that covers 95\% of real-world needs.

Karl Fogel: \emph{Producing Open Source Software: How to Run a
Successful Free Software Project}. O'Reilly Media, 0596007590, 2005.
An guide to how open source projects actually work, full of practical
advice on how to earn commit privileges on a project, get it more
attention, or fork it in case of irreconcilable differences.

Steve Haddock and Casey Dunn: \emph{Practical Computing for
Biologists}. Sinauer, 0878933913, 2010.
An excellent general introduction to ``the other 90\%'' of scientific
computing.

Andy Oram and Greg Wilson (eds): \emph{Making Software: What
Really Works, and Why We Believe It}. O'Reilly, 0596808321, 2010.
Leading software engineering researchers take a chapter each to describe
key empirical results and the evidence behind them. Topics range from
the impact of programming languages on programmers' productivity to
whether we can predict software faults using statistical techniques.

Deborah S. Ray and Eric J. Ray: \emph{Unix and Linux: Visual
QuickStart Guide}. Peachpit Press, 0321636783, 2009.
A gentle introduction to Unix, with many examples.

\subsection{Papers}

Paul F. Dubois: ``Maintaining Correctness in Scientific
Programs''. \emph{Computing in Science \& Engineering}, May--June 2005.
Shows how several good programming practices fit together to create
defense in depth, so that errors missed by one will be caught by
another.

Matthew Gentzkow and Jesse M. Shapiro. 2014: ``Code and Data for
the Social Sciences: A Practitioner's Guide''. University of Chicago
mimeo,
http://faculty.chicagobooth.edu/matthew.gentzkow/research/CodeAndData.pdf,
last updated January 2014.
An excellent description of how to move from doing data analysis with
SAS and Excel to using maintainable scripts on well-organized data files
in a reproducible way.

Jo Erskine Hannay, Hans Petter Langtangen, Carolyn MacLeod,
Dietmar Pfahl, Janice Singer, and Greg Wilson: ``How Do Scientists
Develop and Use Scientific Software?'' \emph{Proc. 2009 ICSE Workshop on
Software Engineering for Computational Science and Engineering}, 2009.
The largest study survey done of how scientists use computers in their
research and how much time they spend doing so.

William Stafford Noble: ``A Quick Guide to Organizing
Computational Biology Projects''. \emph{PLoS Computational Biology},
5(7), 2009.
How and why one scientist organizes his data and scripts.

Ethan P. White, Elita Baldridge, Zachary T. Brym, Kenneth J.
Locey, Daniel J. McGlinn, and Sarah R. Supp: ``Nine Simple Ways to Make
It Easier to (Re)use Your Data.'' \emph{PeerJ PrePrints}, 1:e7v2, 2012.
Delivers exactly what the title promises: a straightforward set of
practices that will make it easier for other scientists to use your
data.

Greg Wilson, D. A. Aruliah, C. Titus Brown, Neil P. Chue Hong,
Matt Davis, Richard T. Guy, Steven H. D. Haddock, Katy Huff, Ian M.
Mitchell, Mark Plumbley, Ben Waugh, Ethan P. White, and Paul Wilson:
``Best Practices for Scientific Computing''. \emph{PLoS Biology}, 12(1),
2014.
Describes a set of best practices for scientific software development
that have solid foundations in research and experience, and that improve
scientists' productivity and the reliability of their software.

Greg Wilson: ``Software Carpentry: Lessons Learned''. \emph{F1000
Research}, 3(62), 2014,
doi:10.12688/f1000research.3-62.v1.
Describes what we've learned about how to teach programming to
scientists over the last 15 years.
